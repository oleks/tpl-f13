\begin{frame}[plain]

\frametitle{Future work}

We may consider binary addition as a ternary bitwise operation with an extra,
carry bit:

\[f : \set{0,1} \times \set{0,1} \times \set{0,1} \rightarrow \set{0,1} \times
\set{0,1},\]

and

\[f\p{x,y,c}=\p{\p{x \oplus y} \oplus c, \p{x \wedge y} \vee \p{ \p{ x \vee y}
\wedge c}}.\]

\end{frame}

\begin{frame}[plain]

\frametitle{Future work}

Recall that we said that a homomorphic ternary function was associative in the
first two arguments iff

\[\forall\ x,y,z,c\in S\ .\ f\p{f\p{x,y,c}, z,c} = f\p{x, f\p{y,z,c},c}.\]

\end{frame}

\begin{frame}[plain]

\frametitle{Future work}

We can apply light's test where we set the carry bit to be first 0, then 1, if
the function is associative in both cases, it is associative in general.

\ \\

Care must be taken when there are multiple carry-based operations in a function
body.

\end{frame}

\begin{frame}[plain]

\frametitle{Future work}

Better \LMIN and \LMAX rules would've also been nice.

\end{frame}
