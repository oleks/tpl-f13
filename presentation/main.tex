\documentclass[12pt, serif, xcolor=table]{beamer}

\usepackage{textpos}
\usepackage{paralist}
\usepackage{enumitem}
\usepackage{expdlist}
\usepackage{subfigure}
\usepackage{amssymb, amsmath, amsthm}
\everymath{\displaystyle}

\usepackage[T1]{fontenc}
%\usepackage{mathpazo}

\definecolor{shade}{RGB}{240,240,240}

%\newcommand{\mono}[1]{\texttt{#1}}
%\newcommand{\textt}[1]{\ensuremath{\text{\mono{#1}}}}
%\newcommand{\mathmono}[1]{\ensuremath{\text{\mono{#1}}}}
%\newcommand{\nonterm}[1]{\ensuremath{\text{\mono{<#1>}}}}
%\newcommand{\term}[1]{\ensuremath{\text{\mono{`#1'}}}}
%\newcommand{\any}[0]{\ensuremath{\left\langle\bigtriangleup\right\rangle}}
%\newcommand{\D}{$\Delta$}

\usepackage{xparse}
\usepackage{survival-pack}
\usepackage{euler}

\newcommand{\makeTable}[6][htbp!]
{
  \begin{table}[#1]
  \centering
  \begin{tabular}{#4}
  #5\\
  #6
  \end{tabular}
  \end{table}
}

\newcommand{\includeFigure}[4][htbp!]
{
  \begin{figure}[#4]
  \centering
  \IfDecimal{#1}
  {
    \includegraphics[scale=#1]{figures/#2}
  }
  {
    \includegraphics[#1]{figures/#2}
  }
  \caption[]{#3}
  \label{figure:#2}
  \end{figure}
}

\usepackage{clrscode3e}
\usepackage{verbatim}
\usepackage{listings}
\lstset
{
  tabsize=2,
  numbers=left,
  breaklines=true,
  foregroundcolor=\color{shade},
  framexleftmargin=0.05in,
  basicstyle=\ttfamily\small,
  numberstyle=\tiny,
%  keywordstyle=\color{green},
%	stringstyle=\color{red},
%	commentstyle=\color{ForestGreen},
  keywords={input},
  mathescape=true,
  captionpos=b,
  escapeinside={(*@}{@*)},
  language=haskell
}

%\beamertemplatenavigationsymbolsempty

\setbeamercolor{title}{fg=shade,bg=black}
\setbeamercolor{frametitle}{fg=shade}
\setbeamercolor{normal text}{fg=shade}
\setbeamercolor{background canvas}{bg=black}
\setbeamercolor{normal text}{bg=black}
\setbeamercolor{bibliography item}{fg=shade}
\setbeamercolor{bibliography entry author}{fg=shade}
\setbeamertemplate{bibliography item}[text]

\renewcommand\tilde[0]{{\raise.17ex\hbox{$\scriptstyle\sim$}}}

\renewcommand{\thefootnote}{\fnsymbol{footnote}}

\newcommand{\bi}{$\bullet$\ }

\newcommand{\lp}[0]{\,\ifmmode \text{\texttt{+}} \else \texttt{+} \fi\,}
\newcommand{\lt}[0]{\,\ifmmode \text{\texttt{*}} \else \texttt{*} \fi\,}
\newcommand{\lm}[0]{\,\ifmmode \text{\texttt{-}} \else \texttt{-} \fi\,}
\newcommand{\ld}[0]{\,\ifmmode \text{\texttt{/}} \else \texttt{/} \fi\,}
\newcommand{\lmod}[0]{\,\ifmmode \text{\texttt{\%}} \else \texttt{\%} \fi\,}
\newcommand{\lpow}[0]{\,\ifmmode \text{\texttt{pow}} \else \texttt{pow} \fi}
\newcommand{\lneg}[0]{{}\ifmmode \text{\tilde} \else \tilde \fi{}}

\newcommand{\LOR}[0]{\,\ifmmode \text{\texttt{||}} \else \texttt{||} \fi\,}
\newcommand{\LAND}[0]{\,\ifmmode \text{\texttt{\&\&}} \else \texttt{\&\&} \fi\,}
\newcommand{\LNOT}[0]{\,\ifmmode \text{\texttt{not}} \else \texttt{not} \fi\,}

\newcommand{\LBXOR}[0]{\,\ifmmode \text{\texttt{\^}} \else \texttt{\^} \fi\,}
\newcommand{\LBAND}[0]{\,\ifmmode \text{\texttt{\&}} \else \texttt{\&} \fi\,}
\newcommand{\LBOR}[0]{\,\ifmmode \text{\texttt{|}} \else \texttt{|} \fi\,}
\newcommand{\LBSL}[0]{\,\ifmmode \text{\texttt{<{}<{}}} \else \texttt{<{}<{}} \fi\,}
\newcommand{\LBSR}[0]{\,\ifmmode \text{\texttt{>{}>{}}} \else \texttt{>{}>{}} \fi\,}

\newcommand{\LCONCAT}[0]{\,\ifmmode \text{\texttt{concat}} \else
\texttt{concat} \fi\,}

\newcommand{\LMIN}[0]{\,\ifmmode \text{\texttt{min}} \else \texttt{min} \fi}
\newcommand{\LMAX}[0]{\,\ifmmode \text{\texttt{max}} \else \texttt{max} \fi}

\title{Deciding associativity in L0}
\author{Kristoffer S\o holm \& Oleksandr Shturmov}
\date{June 19, 2013}

\begin{document}

\begin{frame}

\titlepage

\end{frame}

% Use danish to have a more natural discussion. This may require us to change
% along the way as the concepts become too entangled in english terms.

%\begin{frame}

%\frametitle{Overview}

%\tableofcontents

%\end{frame}

\begin{frame}[plain]

\begin{center}
An infix operator $\oplus : S \times S \rightarrow S$ is associative iff
\end{center}

\[\forall\ x,y,z\in S\ .\ \p{x \oplus y} \oplus z = x \oplus \p{y \oplus z}.\]

\end{frame}

\begin{frame}[plain]

\textbf{Linear reduction}

\ \\

\[\left(\left(\left(\left(\left(\left(x_1\oplus x_2\right) \oplus x_3 \right)
\oplus x_4 \right) x_5 \right) \oplus x_6 \right) \oplus x_7 \right) \oplus
x_8\]

\begin{center}
or
\end{center}

\[x_1 \oplus \left(x_2 \oplus \left(x_3 \oplus \left( x_4 \oplus \left( x_5
\oplus \left( x_6 \oplus \left(x_7\oplus x_8\right) \right) \right) \right)
\right) \right)\]

\end{frame}

\begin{frame}[plain]

\textbf{Tree-like reduction}

\[\big( \left(x_1 \oplus x_2\right) \oplus \left(x_3 \oplus x_4\right) \big)
\oplus \big( \left(x_5 \oplus x_6 \right) \oplus \left(x_7 \oplus x_8\right)
\big)\]

\end{frame}



\begin{frame}[plain]

\textbf{L0}

\ \\

\bi\quad HIPERFIT

\bi\quad Eagerly evaluated.

\bi\quad Purely functional.

\bi\quad Second-order.

\bi\quad Arrays.

\bi\quad Combinators (SOACs).

\bi\quad Efficient execution on vector hardware.

\bi\quad SML-like syntax.

\end{frame}


\begin{frame}[plain,fragile]

\frametitle{Is \texttt{redop} associative?}

\begin{verbatim}
fun (int, int, int, int) redop(
  (int,int,int,int) a, 
  (int,int,int,int) b) =
    let (a0,a1,a2,a3) = a in
    let (b0,b1,b2,b3) = b in
    let mss = max( max(a0, b0), a2 + b1 ) in
    let mis = max( a1, a3 + b1 )          in
    let mcs = max( a2 + b3, b2 )          in
    let ts  = a3 + b3                     in
        (mss, mis, mcs, ts)
\end{verbatim}

\end{frame}


\begin{frame}[plain]

\begin{center}
Let $A$ be a decider for the associative property.
\end{center}

\[M\left(x,y\right)=\left\{
\begin{array}{ll}
x-y\footnotemark & \text{if}\ A(M),\\
z & \text{otherwise}.
\end{array}
\right.\]

\footnotetext{Do something non-associative.}

\end{frame}


\section{The associative type}\label{section:the-associative-type}

For ease of analysis, we introduce the concept of the \emph{associative type}
--- a function type that unifies with the type of any associative function. We
identify this type from the following observation: an associative function
consumes at least two values of some type $S$, and returns a value of type $S$.

We define it formally as follows:

\[\oplus:S \boxtimes S \boxtimes T_1 \boxtimes T_2 \boxtimes \cdots \boxtimes
T_n \rightarrow S, \]

where

\begin{itemize}

\item $\p{T_k}_{k=1}^n$ for $n\geq 0$ is a sequence of arbitrary types.

\item $\boxtimes$ is the commutative equivalent of the regular $\times$, i.e.
the arguments to $\oplus$ can be rearranged under unification.

\end{itemize}

For example, the following types unify with the associative type: $f:S \times S
\times S \rightarrow S$, $g:S \times S \times \mathcal{Z} \rightarrow S$, $h:S
\times \mathcal{Z} \times S \rightarrow S$, etc.

A function with a type that unifies with the associative type is not
necessarily associative, but a function that doesn't definitely is not.

\section{Chaining}

We consider first whether simply chaining applications of an operator is
associative. That is, for each binary operator $\oplus:S\times S \rightarrow
S$, we consider whether $\forall\ x,y,z\in S\ .\ \p{x \oplus y} \oplus z = x
\oplus \p{y \oplus z}$ by the L0 specification.

Unary operators are trivially associative under chaining. For binary operators
with arguments of two different types, as well as ternary and n-ary operators,
it depends. Later, we'll allow intermixing of operators.

Note that even though the binary functions in L0 we are interested in work on arrays, we
will mostly focus on the associativity of binary operators working on single values, and
thus the examples in the text will predominantly be integer based.

\subsubsection{Arithmetic operators}

L0 has a range of binary arithmetic operators, namely \texttt{+}, \texttt{*},
\texttt{-}, \texttt{/}, \texttt{\%}, \texttt{pow}, and \tilde, where the last
operator is a unary negation operator. The operators defined on arguments of
type either \intt{} or \realt{} (L0 is strongly-typed), yielding a value of the
same type.

As of now, there are no restrictions in the L0 standard as to how \realt{}s are
to be represented, so we confine our attention to \intt{}s. For \intt{}s we
assume a $p$-bit two's complement representation.

Our observations are summarized in \referToTable{assoc-chaining-arithop}.

\makeTable
{assoc-chaining-arithop}
{Associativity of chaining arithmetic operators.}
{|c|c|}
{\textbf{Operator} & \textbf{Chaining is associative}}
{
  \texttt{+}   & yes \\
  \texttt{*}   & yes \\
  \texttt{-}   & no \\
  \texttt{/}   & no \\
  \texttt{\%}  & no \\
  \texttt{pow} & no \\
  \tilde       & yes\footnotemark[1]
}

\subsubsection{Relational operators}

L0 has a couple binary relational operators, namely \texttt{<=}, \texttt{<}, and
\texttt{=}, having the obvious semantics. The operators are defined on
arguments both of type either \intt{} or \realt{}, yielding a value
of type \boolt{}.

Relational operators are not chainable in L0. These operators are only
interesting in combination with other operators.\footnotemark[2]

\subsubsection{Logical operators}

L0 has a couple logical operators, namely \texttt{\&\&}, \texttt{||}, and
\texttt{not}, where the last is a unary negation operator. The operators are
defined on arguments of type \boolt{}, yielding a value of type \boolt{}.

Our observations are summarized in \referToTable{assoc-chaining-logop}.

\makeTable
{assoc-chaining-logop}
{Associativity of chaining logical operators.}
{|c|c|}
{\textbf{Operator} & \textbf{Chaining is associative}}
{
  \texttt{\&\&} & yes \\
  \texttt{||}   & yes \\
  \texttt{not}  & yes\footnotemark[1]
}

\footnotetext[1]{Unary operators can be regarded as associative by definition.}

\footnotetext[2]{This is no longer true at the time of our submission.
\texttt{<=}, \texttt{<}, and \texttt{=} are now arithemtic operators, all
left-fixed, and returning integral values.}

\subsubsection{Bitwise operators}

L0 has a range of bitwise operators, namely \texttt{\^}, \texttt{\&},
\texttt{|}, \texttt{>{}>}, and \texttt{<{}<}, having semantics as in C. The
operators are defined on arguments of type \intt{}, yielding a value of type
\intt{}.

Our observations are summarized in \referToTable{assoc-chaining-bitop}.

\makeTable
{assoc-chaining-bitop}
{Associativity of chaining bitwise operators.}
{|c|c|}
{\textbf{Operator} & \textbf{Chaining is associative}}
{
  \texttt{\^} & yes \\
  \texttt{\&} & yes \\
  \texttt{|}  & yes \\
  \texttt{>{}>} & yes \\
  \texttt{<{}<} & yes
}

\subsubsection{Other functions}

L0 has a range of other builtins, e.g. various second-order array combinators.
Of particular interest to us are the functions \texttt{concat}, \texttt{min},
and \texttt{max}, as they are clearly associative.

The other builtins do not meet the requirement of unifiability of their type
with an associative type, or fall prey to the fact that functions are not
first-class citizens in L0.

Our observations are summarized in \referToTable{assoc-array-combinators}.

\makeTable
{assoc-array-combinators}
{Associativity of second-order array combinators}
{|c|c|}
{\textbf{Operator} & \textbf{Chaining is associative}}
{
  \texttt{concat} & yes \\
  \texttt{min} & yes \\
  \texttt{max}  & yes
}

\begin{frame}[plain]

\frametitle{Light's associativity test}

\bi\ Dr. F. W. Light, 1949 \cite{light}.

\bi\ Is a binop $\oplus$, defined in a finite set $S$, associative?

\bi\ For each $a\in S'$, where $S'$ is a generating set of the group
$\left(S,\oplus\right)$, construct the Ca\textbf{y}ley tables of

\begin{align*}
x \circ_{a} y   &= (x \oplus a) \oplus y \\
x *_{a} y       &= x \oplus (a \oplus y) 
\end{align*}

\bi\ If the Ca\textbf{y}ley tables are equivalent, $\oplus$ is associative in
$S$.

\end{frame}

\begin{frame}[plain]

\begin{center}
$x \oplus y = \p{x \LBXOR y} \mid \p{x \LBXOR y}$
\end{center}

\makeTable
{assoc-array-combinators}
{Cayley table for the example operator.}
{|c|c|c|}
{\hline $\oplus$ & $0$ & $1$}
{\hline
    $0$ & $0$ & $1$\\
    $1$ & $1$ & $0$\\\hline
}

\begin{center}

$\left\{0,1\right\}$ is a generating set for the group
$\left(\left\{0,1\right\}, \oplus\right)$\footnote{ $\left\{0\right\}$ clearly
is not.}.

\end{center}

\begin{table}[h!]
  \def\arraystretch{1.2}
  \begin{tabular}{|c|cc|}
    \hline
      $\circ_{0}$ & $0$ & $1$\\
    \hline
      $0$ & $0$ & $1$ \\
      $1$ & $1$ & $0$ \\
    \hline
  \end{tabular}
$=$
  \begin{tabular}{|c|cc|}
    \hline
      $*_{0}$ & $0$ & $1$\\
    \hline
      $0$ & $0$ & $1$ \\
      $1$ & $1$ & $0$ \\
    \hline
  \end{tabular}
\quad
  \begin{tabular}{|c|cc|}
    \hline
      $\circ_{1}$ & $0$ & $1$\\
    \hline
      $0$ & $1$ & $0$ \\
      $1$ & $0$ & $1$ \\
    \hline
  \end{tabular}
$=$
  \begin{tabular}{|c|cc|}
    \hline
      $*_{1}$ & $0$ & $1$\\
    \hline
      $0$ & $1$ & $0$ \\
      $1$ & $0$ & $1$ \\
    \hline
  \end{tabular}

  \centering
\end{table}

\end{frame}

\begin{frame}[plain]

\bi\ Only applicable to combinations of \LBXOR, \LBAND, \LBOR\footnotemark[3].

\bi\ Intractable for e.g. the set of 32-bit integers.

\ \\

\bi\ We may use property-based testing \cite{quickcheck}.

\bi\ Test the property for a sufficient subset of $S$.

\bi\ Seems like the most practical approach.

\bi\ Applicable to the entire language, not just a subset.

\ \\

\bi\ The cost is soundness.

\bi\ The test may generate false positives.

\bi\ We conjecture that a non-associative function will typically be
non-associative for a large subset of $S$.

\footnotetext{Clearly not \LBSL and \LBSR.}

\end{frame}

% L0 is a pure language

\section{Rewriting}

The straight-forward way of testing associativity is by \emph{rewriting}. We
can test whether $\oplus$ is associative by considering whether $\p{x \oplus y}
\oplus z$ and $x \oplus \p{y \oplus z}$ can be rewritten to be syntactically
equivalent.

For instance, if $\oplus\p{x,y} = x\cdot c + y \cdot c$, for some constant
$c\in S$, we consider whether the following equality holds for all $x$, $y$,
and $z$:

\begin{align*}
\p{x\cdot c + y\cdot c}\cdot c + z\cdot c &= x\cdot c + \p{y\cdot c + z\cdot c}\cdot c \\
x\cdot c \cdot c + y\cdot c \cdot c + z\cdot c &= x\cdot c + y\cdot c \cdot c + z\cdot c \cdot c \\
x\cdot c \cdot c + z\cdot c &= x\cdot c + z\cdot c \cdot c
\end{align*}

This analysis required \emph{rewriting} both expressions as a \emph{sum of
products}, followed by \emph{common term elimination}. If $c$ is a constant, we
can complete the analysis by \emph{constant folding}, and considering whether
the left-hand side is syntactically equivalent to the right-hand side.

Note, we didn't need to utilize the properties that addition and multiplication
are also commutative.  This is because the variables appear in the same order
on either side of the equality. So long as our rewriting rules respect this
order, we can proceed freely. 

Another property that our rewriting rules must respect are overflows. We should
keep in mind that we're dealing with modulo arithmetic.

\subsection{Generating variable constraints}

If instead, $\oplus\p{x,y,c} = x\cdot c + y \cdot c$, and we would like to know
whether $\oplus$ is associative wrt. $x$ and $y$, we reach no useful
conclusion. We can find out whether there exists a $c$ for which this property
holds:

\begin{align*}
x \cdot c^2 + z \cdot c &= x \cdot c + z \cdot c^2 \\
x \cdot c + z &= x + z \cdot c \\
x \cdot c - z \cdot c &= x - z \\
c (x-z) &= x-z \\
c &= {x-z\over x-z} \\
c &= 1
\end{align*}

The compiler should at this point suggest that $c$ be replaced by a constant if
the function is to be associative. Such an analysis may not reveal a constant, 

This implies that $\oplus$ is associative for all $x$, $y$, and $z$ only if
$\card{c}=1$. Of course, if $c$ is a constant, it is known to us, and the
equality conclusion can be reached directly without solving for $c$.

To reach this
conclusion, we rewrote both expressions as a sum of products, and found a $c$
such that the left-hand side is syntactically equivalent to the right-hand side
after constant folding.

This technique requires the rewriting of arbitrary program text into some
normal form that subsequently allows equality comparison of program text.

Common subexpression elimination

\begin{frame}[plain]

\frametitle{Future work}

We may consider binary addition as a ternary bitwise operation with an extra,
carry bit:

\[f : \set{0,1} \times \set{0,1} \times \set{0,1} \rightarrow \set{0,1} \times
\set{0,1},\]

and

\[f\p{x,y,c}=\p{\p{x \oplus y} \oplus c, \p{x \wedge y} \vee \p{ \p{ x \vee y}
\wedge c}}.\]

\end{frame}

\begin{frame}[plain]

\frametitle{Future work}

Recall that we said that a homomorphic ternary function was associative in the
first two arguments iff

\[\forall\ x,y,z,c\in S\ .\ f\p{f\p{x,y,c}, z,c} = f\p{x, f\p{y,z,c},c}.\]

\end{frame}

\begin{frame}[plain]

\frametitle{Future work}

We can apply light's test where we set the carry bit to be first 0, then 1, if
the function is associative in both cases, it is associative in general.

\ \\

Care must be taken when there are multiple carry-based operations in a function
body.

\end{frame}

\begin{frame}[plain]

\frametitle{Future work}

Better \LMIN and \LMAX rules would've also been nice.

\end{frame}

\begin{frame}[plain]

\frametitle{Conclusion}

\begin{center}
Use quickcheck.
\end{center}

\end{frame}

\begin{frame}[plain,fragile]

\frametitle{Is \texttt{redop} associative?}

\begin{verbatim}
fun (int, int, int, int) redop(
  (int,int,int,int) a, 
  (int,int,int,int) b) =
    let (a0,a1,a2,a3) = a in
    let (b0,b1,b2,b3) = b in
    let mss = max( max(a0, b0), a2 + b1 ) in
    let mis = max( a1, a3 + b1 )          in
    let mcs = max( a2 + b3, b2 )          in
    let ts  = a3 + b3                     in
        (mss, mis, mcs, ts)
\end{verbatim}

Quickcheck says (most likely) \textbf{yes}.

Rewriting sass (definitely) \textbf{yes} in the last 3 arguments, (most likely)
\textbf{no} in the first argument.

\end{frame}


\begin{thebibliography}{1}

\bibitem[Kehayopulu \& Argyris]{light-mathematica}

Niovi Kehayopulu and Philip Argyris. \emph{An algorithm for Light's
associativity test using Mathematica}. Journal of computing and information,
3(1): 87--98. ISSN 11803886.

\bibitem[Blelloch]{blelloch}

Guy Blelloch. \emph{Scans as Primitive Parallel Operations}. 1987. IEEE
Transactions on Computers. Volume 38. 1526--1538.

\bibitem[A.H. Clifford \& G.B. Preston]{light}
Clifford, A.H. and Preston, G.B.,
\emph{The Algebraic Theory of Semigroups, volume 1}. 1961.
lc61015686i. \url{http://books.google.dk/books?id=Ib0mAAAAMAAJ}.
  American Mathematical Society. 1--9.

\bibitem[QuickCheck]{quickcheck}
\url{http://www.cse.chalmers.se/~rjmh/QuickCheck/}

\end{thebibliography}


\end{document}
