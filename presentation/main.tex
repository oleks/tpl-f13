\documentclass[12pt, serif, xcolor=table]{beamer}

\usepackage{textpos}
\usepackage{paralist}
\usepackage{enumitem}
\usepackage{expdlist}
\usepackage{subfigure}
\usepackage{amssymb, amsmath, amsthm}
\everymath{\displaystyle}

\usepackage[T1]{fontenc}
%\usepackage{mathpazo}

\definecolor{shade}{RGB}{240,240,240}

%\newcommand{\mono}[1]{\texttt{#1}}
%\newcommand{\textt}[1]{\ensuremath{\text{\mono{#1}}}}
%\newcommand{\mathmono}[1]{\ensuremath{\text{\mono{#1}}}}
%\newcommand{\nonterm}[1]{\ensuremath{\text{\mono{<#1>}}}}
%\newcommand{\term}[1]{\ensuremath{\text{\mono{`#1'}}}}
%\newcommand{\any}[0]{\ensuremath{\left\langle\bigtriangleup\right\rangle}}
%\newcommand{\D}{$\Delta$}

\usepackage{xparse}
\usepackage{survival-pack}
\usepackage{euler}

\newcommand{\makeTable}[6][htbp!]
{
  \begin{table}[#1]
  \centering
  \begin{tabular}{#4}
  #5\\
  #6
  \end{tabular}
  \end{table}
}

\newcommand{\includeFigure}[4][htbp!]
{
  \begin{figure}[#4]
  \centering
  \IfDecimal{#1}
  {
    \includegraphics[scale=#1]{figures/#2}
  }
  {
    \includegraphics[#1]{figures/#2}
  }
  \caption[]{#3}
  \label{figure:#2}
  \end{figure}
}

\usepackage{clrscode3e}
\usepackage{verbatim}
\usepackage{listings}
\lstset
{
  tabsize=2,
  numbers=left,
  breaklines=true,
  foregroundcolor=\color{shade},
  framexleftmargin=0.05in,
  basicstyle=\ttfamily\small,
  numberstyle=\tiny,
%  keywordstyle=\color{green},
%	stringstyle=\color{red},
%	commentstyle=\color{ForestGreen},
  keywords={input},
  mathescape=true,
  captionpos=b,
  escapeinside={(*@}{@*)},
  language=haskell
}

%\beamertemplatenavigationsymbolsempty

\setbeamercolor{title}{fg=shade,bg=black}
\setbeamercolor{frametitle}{fg=shade}
\setbeamercolor{normal text}{fg=shade}
\setbeamercolor{background canvas}{bg=black}
\setbeamercolor{normal text}{bg=black}
\setbeamercolor{bibliography item}{fg=shade}
\setbeamercolor{bibliography entry author}{fg=shade}
\setbeamertemplate{bibliography item}[text]

\renewcommand\tilde[0]{{\raise.17ex\hbox{$\scriptstyle\sim$}}}

\renewcommand{\thefootnote}{\fnsymbol{footnote}}

\newcommand{\bi}{$\bullet$\ }

\newcommand{\lp}[0]{\,\ifmmode \text{\texttt{+}} \else \texttt{+} \fi\,}
\newcommand{\lt}[0]{\,\ifmmode \text{\texttt{*}} \else \texttt{*} \fi\,}
\newcommand{\lm}[0]{\,\ifmmode \text{\texttt{-}} \else \texttt{-} \fi\,}
\newcommand{\ld}[0]{\,\ifmmode \text{\texttt{/}} \else \texttt{/} \fi\,}
\newcommand{\lmod}[0]{\,\ifmmode \text{\texttt{\%}} \else \texttt{\%} \fi\,}
\newcommand{\lpow}[0]{\,\ifmmode \text{\texttt{pow}} \else \texttt{pow} \fi}
\newcommand{\lneg}[0]{{}\ifmmode \text{\tilde} \else \tilde \fi{}}

\newcommand{\LOR}[0]{\,\ifmmode \text{\texttt{||}} \else \texttt{||} \fi\,}
\newcommand{\LAND}[0]{\,\ifmmode \text{\texttt{\&\&}} \else \texttt{\&\&} \fi\,}
\newcommand{\LNOT}[0]{\,\ifmmode \text{\texttt{not}} \else \texttt{not} \fi\,}

\newcommand{\LBXOR}[0]{\,\ifmmode \text{\texttt{\^}} \else \texttt{\^} \fi\,}
\newcommand{\LBAND}[0]{\,\ifmmode \text{\texttt{\&}} \else \texttt{\&} \fi\,}
\newcommand{\LBOR}[0]{\,\ifmmode \text{\texttt{|}} \else \texttt{|} \fi\,}
\newcommand{\LBSL}[0]{\,\ifmmode \text{\texttt{<{}<{}}} \else \texttt{<{}<{}} \fi\,}
\newcommand{\LBSR}[0]{\,\ifmmode \text{\texttt{>{}>{}}} \else \texttt{>{}>{}} \fi\,}

\newcommand{\LCONCAT}[0]{\,\ifmmode \text{\texttt{concat}} \else
\texttt{concat} \fi\,}

\newcommand{\LMIN}[0]{\,\ifmmode \text{\texttt{min}} \else \texttt{min} \fi}
\newcommand{\LMAX}[0]{\,\ifmmode \text{\texttt{max}} \else \texttt{max} \fi}

\title{Deciding associativity in L0}
\author{Kristoffer S\o holm \& Oleksandr Shturmov}
\date{June 19, 2013}

\begin{document}

\begin{frame}

\titlepage

\end{frame}

% Use danish to have a more natural discussion. This may require us to change
% along the way as the concepts become too entangled in english terms.

%\begin{frame}

%\frametitle{Overview}

%\tableofcontents

%\end{frame}

\begin{frame}[plain]

\begin{center}
An infix operator $\oplus : S \times S \rightarrow S$ is associative iff
\end{center}

\[\forall\ x,y,z\in S\ .\ \p{x \oplus y} \oplus z = x \oplus \p{y \oplus z}.\]

\end{frame}

\begin{frame}[plain]

\textbf{Linear reduction}

\ \\

\[\left(\left(\left(\left(\left(\left(x_1\oplus x_2\right) \oplus x_3 \right)
\oplus x_4 \right) x_5 \right) \oplus x_6 \right) \oplus x_7 \right) \oplus
x_8\]

\begin{center}
or
\end{center}

\[x_1 \oplus \left(x_2 \oplus \left(x_3 \oplus \left( x_4 \oplus \left( x_5
\oplus \left( x_6 \oplus \left(x_7\oplus x_8\right) \right) \right) \right)
\right) \right)\]

\end{frame}

\begin{frame}[plain]

\textbf{Tree-like reduction}

\[\big( \left(x_1 \oplus x_2\right) \oplus \left(x_3 \oplus x_4\right) \big)
\oplus \big( \left(x_5 \oplus x_6 \right) \oplus \left(x_7 \oplus x_8\right)
\big)\]

\end{frame}



\begin{frame}[plain]

\textbf{L0}

\ \\

\bi\quad HIPERFIT

\bi\quad Eagerly evaluated.

\bi\quad Purely functional.

\bi\quad Second-order.

\bi\quad Arrays.

\bi\quad Combinators (SOACs).

\bi\quad Efficient execution on vector hardware.

\bi\quad SML-like syntax.

\end{frame}


\begin{frame}[plain,fragile]

\frametitle{Is \texttt{redop} associative?}

\begin{verbatim}
fun (int, int, int, int) redop(
  (int,int,int,int) a, 
  (int,int,int,int) b) =
    let (a0,a1,a2,a3) = a in
    let (b0,b1,b2,b3) = b in
    let mss = max( max(a0, b0), a2 + b1 ) in
    let mis = max( a1, a3 + b1 )          in
    let mcs = max( a2 + b3, b2 )          in
    let ts  = a3 + b3                     in
        (mss, mis, mcs, ts)
\end{verbatim}

\end{frame}


\section{Undecidability}\label{section:undecidability}

Undecidability of the property is proven similar to the halting problem.

\begin{proof} We construct a binary Turing machine (TM) $M$ that takes as input
a pair of values $x,y\in S$.

Assume there exists a TM $A$, that given an binary TM $M$, accepts if $M$ is
associative, and rejects otherwise. That is, $A$ is a decider for the
associativity problem.

We construct $M$ such that it returns $x$ if $A(M)$ accepts, and a constant
$z\in S$ otherwise:

\[M\p{x,y} = \left\{ \begin{array}{l l} x & \text{if $A(M)$} \\ z &
\text{otherwise} \end{array} \right.\]

If $A(M)$ accepts, $M$ is not associative, if $A(M)$ rejects, $M$ is
associative, which contradicts our assumptions about $A$. \end{proof}

The associative property is Turing-recognizable however, as we can check
whether the associative property holds for $\oplus$ for all possible $x$, $y$,
and $z$. Our approach is to choose some subset of L0 for which we can decide
associativity within reasonable time.


\begin{frame}[plain]

\frametitle{The associative type}

A function type that unifies with all associative functions.

\[S \boxtimes S \boxtimes T_1 \boxtimes T_2 \boxtimes \cdots \boxtimes T_n
\rightarrow S, \]

For example, $f:S \times S \times S \rightarrow S$.

We say that $f$ is associative in its first two arguments iff

\[\forall\ x,y,z,c\in S\ .\ f\left(x,f\left(y,z,c\right),c\right) =
f\left(\left( x, y, c\right), z, c\right).\]

\end{frame}

\section{Chaining}

We consider first whether simply chaining applications of an operator is
associative. That is, for each binary operator $\oplus:S\times S \rightarrow
S$, we consider whether $\forall\ x,y,z\in S\ .\ \p{x \oplus y} \oplus z = x
\oplus \p{y \oplus z}$ by the L0 specification.

Unary operators are trivially associative under chaining. For binary operators
with arguments of two different types, as well as ternary and n-ary operators,
it depends. Later, we'll allow intermixing of operators.

Note that even though the binary functions in L0 we are interested in work on arrays, we
will mostly focus on the associativity of binary operators working on single values, and
thus the examples in the text will predominantly be integer based.

\subsubsection{Arithmetic operators}

L0 has a range of binary arithmetic operators, namely \texttt{+}, \texttt{*},
\texttt{-}, \texttt{/}, \texttt{\%}, \texttt{pow}, and \tilde, where the last
operator is a unary negation operator. The operators defined on arguments of
type either \intt{} or \realt{} (L0 is strongly-typed), yielding a value of the
same type.

As of now, there are no restrictions in the L0 standard as to how \realt{}s are
to be represented, so we confine our attention to \intt{}s. For \intt{}s we
assume a $p$-bit two's complement representation.

Our observations are summarized in \referToTable{assoc-chaining-arithop}.

\makeTable
{assoc-chaining-arithop}
{Associativity of chaining arithmetic operators.}
{|c|c|}
{\textbf{Operator} & \textbf{Chaining is associative}}
{
  \texttt{+}   & yes \\
  \texttt{*}   & yes \\
  \texttt{-}   & no \\
  \texttt{/}   & no \\
  \texttt{\%}  & no \\
  \texttt{pow} & no \\
  \tilde       & yes\footnotemark[1]
}

\subsubsection{Relational operators}

L0 has a couple binary relational operators, namely \texttt{<=}, \texttt{<}, and
\texttt{=}, having the obvious semantics. The operators are defined on
arguments both of type either \intt{} or \realt{}, yielding a value
of type \boolt{}.

Relational operators are not chainable in L0. These operators are only
interesting in combination with other operators.\footnotemark[2]

\subsubsection{Logical operators}

L0 has a couple logical operators, namely \texttt{\&\&}, \texttt{||}, and
\texttt{not}, where the last is a unary negation operator. The operators are
defined on arguments of type \boolt{}, yielding a value of type \boolt{}.

Our observations are summarized in \referToTable{assoc-chaining-logop}.

\makeTable
{assoc-chaining-logop}
{Associativity of chaining logical operators.}
{|c|c|}
{\textbf{Operator} & \textbf{Chaining is associative}}
{
  \texttt{\&\&} & yes \\
  \texttt{||}   & yes \\
  \texttt{not}  & yes\footnotemark[1]
}

\footnotetext[1]{Unary operators can be regarded as associative by definition.}

\footnotetext[2]{This is no longer true at the time of our submission.
\texttt{<=}, \texttt{<}, and \texttt{=} are now arithemtic operators, all
left-fixed, and returning integral values.}

\subsubsection{Bitwise operators}

L0 has a range of bitwise operators, namely \texttt{\^}, \texttt{\&},
\texttt{|}, \texttt{>{}>}, and \texttt{<{}<}, having semantics as in C. The
operators are defined on arguments of type \intt{}, yielding a value of type
\intt{}.

Our observations are summarized in \referToTable{assoc-chaining-bitop}.

\makeTable
{assoc-chaining-bitop}
{Associativity of chaining bitwise operators.}
{|c|c|}
{\textbf{Operator} & \textbf{Chaining is associative}}
{
  \texttt{\^} & yes \\
  \texttt{\&} & yes \\
  \texttt{|}  & yes \\
  \texttt{>{}>} & yes \\
  \texttt{<{}<} & yes
}

\subsubsection{Other functions}

L0 has a range of other builtins, e.g. various second-order array combinators.
Of particular interest to us are the functions \texttt{concat}, \texttt{min},
and \texttt{max}, as they are clearly associative.

The other builtins do not meet the requirement of unifiability of their type
with an associative type, or fall prey to the fact that functions are not
first-class citizens in L0.

Our observations are summarized in \referToTable{assoc-array-combinators}.

\makeTable
{assoc-array-combinators}
{Associativity of second-order array combinators}
{|c|c|}
{\textbf{Operator} & \textbf{Chaining is associative}}
{
  \texttt{concat} & yes \\
  \texttt{min} & yes \\
  \texttt{max}  & yes
}

\begin{frame}[plain]

\frametitle{Light's associativity test}

\bi\ Dr. F. W. Light, 1949 \cite{light}.

\bi\ Is a binop $\oplus$, defined in a finite set $S$, associative?

\bi\ For each $a\in S'$, where $S'$ is a generating set of the group
$\left(S,\oplus\right)$, construct the Ca\textbf{y}ley tables of

\begin{align*}
x \circ_{a} y   &= (x \oplus a) \oplus y \\
x *_{a} y       &= x \oplus (a \oplus y) 
\end{align*}

\bi\ If the Ca\textbf{y}ley tables are equivalent, $\oplus$ is associative in
$S$.

\end{frame}

\begin{frame}[plain]

\begin{center}
$x \oplus y = \p{x \LBXOR y} \mid \p{x \LBXOR y}$
\end{center}

\makeTable
{assoc-array-combinators}
{Cayley table for the example operator.}
{|c|c|c|}
{\hline $\oplus$ & $0$ & $1$}
{\hline
    $0$ & $0$ & $1$\\
    $1$ & $1$ & $0$\\\hline
}

\begin{center}

$\left\{0,1\right\}$ is a generating set for the group
$\left(\left\{0,1\right\}, \oplus\right)$\footnote{ $\left\{0\right\}$ clearly
is not.}.

\end{center}

\begin{table}[h!]
  \def\arraystretch{1.2}
  \begin{tabular}{|c|cc|}
    \hline
      $\circ_{0}$ & $0$ & $1$\\
    \hline
      $0$ & $0$ & $1$ \\
      $1$ & $1$ & $0$ \\
    \hline
  \end{tabular}
$=$
  \begin{tabular}{|c|cc|}
    \hline
      $*_{0}$ & $0$ & $1$\\
    \hline
      $0$ & $0$ & $1$ \\
      $1$ & $1$ & $0$ \\
    \hline
  \end{tabular}
\quad
  \begin{tabular}{|c|cc|}
    \hline
      $\circ_{1}$ & $0$ & $1$\\
    \hline
      $0$ & $1$ & $0$ \\
      $1$ & $0$ & $1$ \\
    \hline
  \end{tabular}
$=$
  \begin{tabular}{|c|cc|}
    \hline
      $*_{1}$ & $0$ & $1$\\
    \hline
      $0$ & $1$ & $0$ \\
      $1$ & $0$ & $1$ \\
    \hline
  \end{tabular}

  \centering
\end{table}

\end{frame}

\begin{frame}[plain]

\bi\ Only applicable to combinations of \LBXOR, \LBAND, \LBOR\footnotemark[3].

\bi\ Intractable for e.g. the set of 32-bit integers.

\ \\

\bi\ We may use property-based testing \cite{quickcheck}.

\bi\ Test the property for a sufficient subset of $S$.

\bi\ Seems like the most practical approach.

\bi\ Applicable to the entire language, not just a subset.

\ \\

\bi\ The cost is soundness.

\bi\ The test may generate false positives.

\bi\ We conjecture that a non-associative function will typically be
non-associative for a large subset of $S$.

\footnotetext{Clearly not \LBSL and \LBSR.}

\end{frame}

\begin{frame}[plain]

\frametitle{Rewriting}

Consider

\[\left(x\oplus y\right)\oplus z = x \oplus \left(y \oplus z\right).\]

Expand all applications of $\oplus$\footnotemark[4] and see if the expressions
on either side of the equality are semantically equivalent.

\footnotetext[4]{Take care not to recurse.}

\end{frame}

\begin{frame}[plain]

\frametitle{Rewriting}

Consider

\[\left(x\oplus y\right)\oplus z = x \oplus \left(y \oplus z\right).\]

This can be done by rewriting the expressions until they are semantically
equivalent, or until we've run out of rules.

\ \\

Note, the variables appear in the same order on either side of the equality. If
the rewriting rules respect this order, there is no need for normalization.

\ \\

The rewriting rules must also respect overflow properties\footnotemark[5].

\footnotetext[5]{They don't.}

\end{frame}

\begin{frame}

\frametitle{Rewrite to be left-associative}

\footnotesize
\[
\begin{matrix}
\proc{Plus}:{\over
e \lp \p{e_1 \lp e_2} \rightarrow \p{ e \lp e_1} \lp e_2
}\\
\proc{Times}:{\over
e \lt \p{e_1\lt e_2} \rightarrow \p{e \lt e_1} \lt e_2
}\\
\proc{Land}:{\over
e \LAND \p{e_1 \LAND e_2} \rightarrow \p{ e \LAND e_1} \LAND e_2
}\\
\proc{Lor}:{\over
e \LOR \p{e_1\LOR e_2} \rightarrow \p{e \LOR e_1} \LOR e_2
}\\
\proc{Xor}:{\over
e \LBXOR \p{e_1 \LBXOR e_2} \rightarrow \p{ e \LBXOR e_1} \LBXOR e_2
}\\
\proc{Band}:{\over
e \LBAND \p{e_1 \LAND e_2} \rightarrow \p{ e \LBAND e_1} \LBAND e_2
}\\
\proc{Bor}:{\over
e \LBOR \p{e_1\LBOR e_2} \rightarrow \p{e \LBOR e_1} \LBOR e_2
}\\
\proc{BSL}:{\over
e \LBSL \p{e_1 \LBSL e_2} \rightarrow \p{ e \LBSL e_1} \LBSL e_2
}\\
\proc{BSR}:{\over
e \LBSR \p{e_1\LBSR e_2} \rightarrow \p{e \LBSR e_1} \LBSR e_2
}\\
\proc{Min}:{\over
\LMIN\p{e, \LMIN\p{e_1, e_2}} \rightarrow \LMIN\p{\LMIN\p{ e, e_1}, e_2}
}\\
\proc{Min}:{\over
\LMAX\p{e, \LMAX\p{e_1, e_2}} \rightarrow \LMAX\p{\LMAX\p{ e, e_1}, e_2}
}\\
\proc{Concat}:{\over
\LCONCAT\p{e, \LCONCAT\p{e_1, e_2}} \rightarrow \LCONCAT\p{\LCONCAT\p{ e, e_1}, e_2}
}\\
\end{matrix}
\]

\end{frame}

\begin{frame}[plain]

\frametitle{Left-distributive laws}

\footnotesize
\[
\begin{matrix}
\proc{P-Minus-L}:{\over
e \lp \p{e_1 \lm e_2} \rightarrow \p{e \lp e_1 } \lm e_2
}\\
\proc{P-Min-L}:{\over
e \lp \LMIN\p{e_1, e_2} \rightarrow \LMIN\p{\p{e \lp e_1}, \p{e \lp e_2}}
}\footnotemark[5]\\
\proc{P-Max-L}:{\over
e \lp \LMAX\p{e_1, e_2} \rightarrow \LMAX\p{\p{e \lp e_1}, \p{e \lp e_2}}
}\footnotemark[5]\\
\proc{T-Plus-L}:{\over
e \lt \p{e_1 \lp e_2} \rightarrow \p{ e \lt e_1 } \lp \p{ e \lt e_2 }
}\\
\proc{T-Minus-L}:{\over
e \lt \p{e_1 \lm e_2} \rightarrow \p{ e\lt e_1 } \lm \p{ e \lt e_2 }
}\\
\proc{T-Mod-L}:{\over
e \lt \p{e_1 \lmod e_2}\rightarrow \p{ e \lt e_1 } \lmod \p{ e \lt e_2 }
}\\
\proc{T-ShiftL-L}:{\over
e \lt \p{e_1 \LBSL e_2} \rightarrow \p{ e \lt e_1 } \LBSL e_2
}\\
\proc{T-Min-L}:{\over
e \lt \LMIN\p{e_1, e_2} \rightarrow \LMIN\p{\p{e \lt e_1}, \p{e \lt e_2}}
}\footnotemark[5]\\
\proc{T-Max-L}:{\over
e \lt \LMAX\p{e_1, e_2} \rightarrow \LMAX\p{\p{e \lt e_1}, \p{e \lt e_2}}
}\footnotemark[5]\\
\proc{A-Or-L}:{\over
e \LAND \p{e_1 \LOR e_2}\rightarrow \p{ e \LAND e_1 } \LOR \p{ e \LAND e_2 }
}\\
\proc{BA-BOr-L}:{\over
e \LBAND \p{e_1 \LBOR e_2}\rightarrow \p{ e \LBAND e_1 } \LBOR \p{ e \LBAND e_2
}
}
\end{matrix}
\]

\end{frame}

\begin{frame}

\frametitle{Proofs?}

By structural recursion on the semantics of L0\footnotemark[6].

\footnotetext[6]{Which is not formally specified.}

\end{frame}


\section{Future work}

One possible direction of future work is to expand Light's test beyond bitwise
operators. We can make the observation that a binary addition is a ternary
bitwise operator, where for every pair of bits there is a third, carry bit.

We can state addition as a bitwise operation, i.e. $f : \set{0,1} \times
\set{0,1} \times \set{0,1} \rightarrow \set{0,1} \times \set{0,1}$, and
$f\p{x,y,c}=\p{\p{x \oplus y} \oplus c, \p{x \wedge y} \vee \p{ \p{ x \vee y}
\wedge c}}$, where the first element is the result bit, and the second element
is the carry bit. Note, a function like $f$ in L0, using bitwise operators over
integers, is by no means the same as addition over integers.

If we apply the idea of the associative type, we can state the problem of
testing the associativity of integer addition as testing whether for all
$x,y,z,c\in \set{0,1}$:

\[f\p{f\p{x,y,c}, z,c} = f\p{x, f\p{y,z,c},c}.\]

Light's associativity can be applied here by first letting $c=0$ and then
letting $c=1$. If $f$ turns out to be associative in both these cases, we can
say that $f$ is associative in general. 

Problems arise with more than one addition in a single expression, as this requires more
than a single bit for the carry bit, but which might be solvable by expanding the Caley
tables to include the combined carry as an integer.

This approach requires time linear in the number of carry-based operations in
the function body, i.e. there is an independent carry bit for each such
operation. This makes the problem feasible such operations, unlike considering
the entire integer domain.

Subtraction can be dealt with in a similar manner, if we keep in mind that
$x\lp y=x \lp(\lneg y)$. Other operators however, get more tricky as we have to
keep more than a single carry bit in mind when doing e.g. multiplication,
division, raising to various powers, or bit shifting.

Another possible direction of future work is expanding the set of rewriting
rules. Especially useful would be generalised rules about \texttt{min},
\texttt{max}, and \texttt{if-then-else}.

A third, perhaps less useful direction is to generate variable constraints for
unconstrained variables if the function were to be made associative.

If $\oplus\p{x,y,c} = \p{c\lt x} \lp \p{c \lt y}$, and we would like to know
whether $\oplus$ is associative wrt. $x$ and $y$, we reach no useful conclusion
with the above method. We can find out whether there exists a $c$ for which
this property holds:

\begin{align*}
\p{\p{c \lt c} \lt x} \lp \p{c \lt z} &= \p{c \lt x} \lp \p{\p{c \lt c} \lt z} \\
\p{c \lt \p{c \lt x}} \lp \p{c \lt z} &= \p{c \lt x} \lp \p{c \lt \p{c \lt z}} \\
\p{c \lt x} \lp z &= x \lp \p{c \lt z} \\
\p{c \lt x} \lm \p{c \lt z} &= x \lm z \\
c \lt \p{x \lm z} &= x \lm z \\
c &= \p{x \lm z} \ld  \p{x \lm z} \\
c &= 1
\end{align*}

The compiler could at this point suggest that $c$ be replaced by a constant if
the function is to be associative. Such an analysis may reveal a constant, a
range, or a set of possible values, or it may fail completely.


\section{Conclusion}

The undecidability of the associative property puts a bound on it's
theoretical, but not necessarily practical feasibility.

We have looked into mutliple methods for deciding associativity. Only one of
these methods is applicable to L0 programs in general. The cost is loss of
soundness, i.e. functions may be deemed associative when they in fact ar not.
Other methods are sound (no false positives), but constrained to a rather
mundane subset of the L0 language. These methods also do not in general allow
for the function in question to use other user-defined functions, unless these
functions can be efficiently inlined.

The techniques presented in this paper are by no means constrained to L0. They
are aplicable to all languages with similar semantics.


\begin{frame}

\frametitle{References\footnotemark[7]}

\begin{thebibliography}{9}

\bibitem[Blelloch]{blelloch}

Guy Blelloch. \emph{Scans as Primitive Parallel Operations}. 1987. IEEE
Transactions on Computers. Volume 38. 1526--1538.

\bibitem[Clifford \& Preston]{light}

Alfred Hoblitzelle Clifford and Gordon Bambord Preston. \emph{The Algebraic
Theory of Semigroups, Volume 1}. 1961. American Mathematical Society. 1--9.
URL: \url{http://books.google.dk/books?id=Ib0mAAAAMAAJ}.

\bibitem[Claessen \& Hughes]{quickcheck}

Koen Claessen and John Hughes. \emph{QuickCheck: a lightweight tool for random
testing of Haskell programs}. 2000. SIGPLAN Not. 35, 9. 268--279.

\end{thebibliography}

\footnotetext[7]{A dozen more would've been nice.}

\end{frame}


\end{document}
