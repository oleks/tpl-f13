\begin{frame}[plain]

\frametitle{Rewriting}

Consider

\[\left(x\oplus y\right)\oplus z = x \oplus \left(y \oplus z\right).\]

Expand all applications of $\oplus$\footnotemark[4] and see if the expressions
on either side of the equality are semantically equivalent.

\footnotetext[4]{Take care not to recurse.}

\end{frame}

\begin{frame}[plain]

\frametitle{Rewriting}

Consider

\[\left(x\oplus y\right)\oplus z = x \oplus \left(y \oplus z\right).\]

This can be done by rewriting the expressions until they are semantically
equivalent, or until we've run out of rules.

\ \\

Note, the variables appear in the same order on either side of the equality. If
the rewriting rules respect this order, there is no need for normalization.

\ \\

The rewriting rules must also respect overflow properties\footnotemark[5].

\footnotetext[5]{They don't.}

\end{frame}

\begin{frame}

\frametitle{Rewrite to be left-associative}

\footnotesize
\[
\begin{matrix}
\proc{Plus}:{\over
e \lp \p{e_1 \lp e_2} \rightarrow \p{ e \lp e_1} \lp e_2
}\\
\proc{Times}:{\over
e \lt \p{e_1\lt e_2} \rightarrow \p{e \lt e_1} \lt e_2
}\\
\proc{Land}:{\over
e \LAND \p{e_1 \LAND e_2} \rightarrow \p{ e \LAND e_1} \LAND e_2
}\\
\proc{Lor}:{\over
e \LOR \p{e_1\LOR e_2} \rightarrow \p{e \LOR e_1} \LOR e_2
}\\
\proc{Xor}:{\over
e \LBXOR \p{e_1 \LBXOR e_2} \rightarrow \p{ e \LBXOR e_1} \LBXOR e_2
}\\
\proc{Band}:{\over
e \LBAND \p{e_1 \LAND e_2} \rightarrow \p{ e \LBAND e_1} \LBAND e_2
}\\
\proc{Bor}:{\over
e \LBOR \p{e_1\LBOR e_2} \rightarrow \p{e \LBOR e_1} \LBOR e_2
}\\
\proc{BSL}:{\over
e \LBSL \p{e_1 \LBSL e_2} \rightarrow \p{ e \LBSL e_1} \LBSL e_2
}\\
\proc{BSR}:{\over
e \LBSR \p{e_1\LBSR e_2} \rightarrow \p{e \LBSR e_1} \LBSR e_2
}\\
\proc{Min}:{\over
\LMIN\p{e, \LMIN\p{e_1, e_2}} \rightarrow \LMIN\p{\LMIN\p{ e, e_1}, e_2}
}\\
\proc{Min}:{\over
\LMAX\p{e, \LMAX\p{e_1, e_2}} \rightarrow \LMAX\p{\LMAX\p{ e, e_1}, e_2}
}\\
\proc{Concat}:{\over
\LCONCAT\p{e, \LCONCAT\p{e_1, e_2}} \rightarrow \LCONCAT\p{\LCONCAT\p{ e, e_1}, e_2}
}\\
\end{matrix}
\]

\end{frame}

\begin{frame}[plain]

\frametitle{Left-distributive laws}

\footnotesize
\[
\begin{matrix}
\proc{P-Minus-L}:{\over
e \lp \p{e_1 \lm e_2} \rightarrow \p{e \lp e_1 } \lm e_2
}\\
\proc{P-Min-L}:{\over
e \lp \LMIN\p{e_1, e_2} \rightarrow \LMIN\p{\p{e \lp e_1}, \p{e \lp e_2}}
}\footnotemark[5]\\
\proc{P-Max-L}:{\over
e \lp \LMAX\p{e_1, e_2} \rightarrow \LMAX\p{\p{e \lp e_1}, \p{e \lp e_2}}
}\footnotemark[5]\\
\proc{T-Plus-L}:{\over
e \lt \p{e_1 \lp e_2} \rightarrow \p{ e \lt e_1 } \lp \p{ e \lt e_2 }
}\\
\proc{T-Minus-L}:{\over
e \lt \p{e_1 \lm e_2} \rightarrow \p{ e\lt e_1 } \lm \p{ e \lt e_2 }
}\\
\proc{T-Mod-L}:{\over
e \lt \p{e_1 \lmod e_2}\rightarrow \p{ e \lt e_1 } \lmod \p{ e \lt e_2 }
}\\
\proc{T-ShiftL-L}:{\over
e \lt \p{e_1 \LBSL e_2} \rightarrow \p{ e \lt e_1 } \LBSL e_2
}\\
\proc{T-Min-L}:{\over
e \lt \LMIN\p{e_1, e_2} \rightarrow \LMIN\p{\p{e \lt e_1}, \p{e \lt e_2}}
}\footnotemark[5]\\
\proc{T-Max-L}:{\over
e \lt \LMAX\p{e_1, e_2} \rightarrow \LMAX\p{\p{e \lt e_1}, \p{e \lt e_2}}
}\footnotemark[5]\\
\proc{A-Or-L}:{\over
e \LAND \p{e_1 \LOR e_2}\rightarrow \p{ e \LAND e_1 } \LOR \p{ e \LAND e_2 }
}\\
\proc{BA-BOr-L}:{\over
e \LBAND \p{e_1 \LBOR e_2}\rightarrow \p{ e \LBAND e_1 } \LBOR \p{ e \LBAND e_2
}
}
\end{matrix}
\]

\end{frame}

\begin{frame}

\frametitle{Proofs?}

By structural recursion on the semantics of L0\footnotemark[6].

\footnotetext[6]{Which is not formally specified.}

\end{frame}

