\begin{frame}[plain]

\frametitle{Light's associativity test}

\bi\ Dr. F. W. Light, 1949 \cite{light}.

\bi\ Is a binop $\oplus$, defined in a finite set $S$, associative?

\bi\ For each $a\in S'$, where $S'$ is a generating set of the group
$\left(S,\oplus\right)$, construct the Ca\textbf{y}ley tables of

\begin{align*}
x \circ_{a} y   &= (x \oplus a) \oplus y \\
x *_{a} y       &= x \oplus (a \oplus y) 
\end{align*}

\bi\ If the Ca\textbf{y}ley tables are equivalent, $\oplus$ is associative in
$S$.

\end{frame}

\begin{frame}[plain]

\begin{center}
$x \oplus y = \p{x \LBXOR y} \mid \p{x \LBXOR y}$
\end{center}

\makeTable
{assoc-array-combinators}
{Cayley table for the example operator.}
{|c|c|c|}
{\hline $\oplus$ & $0$ & $1$}
{\hline
    $0$ & $0$ & $1$\\
    $1$ & $1$ & $0$\\\hline
}

\begin{center}

$\left\{0,1\right\}$ is a generating set for the group
$\left(\left\{0,1\right\}, \oplus\right)$\footnote{ $\left\{0\right\}$ clearly
is not.}.

\end{center}

\begin{table}[h!]
  \def\arraystretch{1.2}
  \begin{tabular}{|c|cc|}
    \hline
      $\circ_{0}$ & $0$ & $1$\\
    \hline
      $0$ & $0$ & $1$ \\
      $1$ & $1$ & $0$ \\
    \hline
  \end{tabular}
$=$
  \begin{tabular}{|c|cc|}
    \hline
      $*_{0}$ & $0$ & $1$\\
    \hline
      $0$ & $0$ & $1$ \\
      $1$ & $1$ & $0$ \\
    \hline
  \end{tabular}
\quad
  \begin{tabular}{|c|cc|}
    \hline
      $\circ_{1}$ & $0$ & $1$\\
    \hline
      $0$ & $1$ & $0$ \\
      $1$ & $0$ & $1$ \\
    \hline
  \end{tabular}
$=$
  \begin{tabular}{|c|cc|}
    \hline
      $*_{1}$ & $0$ & $1$\\
    \hline
      $0$ & $1$ & $0$ \\
      $1$ & $0$ & $1$ \\
    \hline
  \end{tabular}

  \centering
\end{table}

\end{frame}

\begin{frame}[plain]

\bi\ Only applicable to combinations of \LBXOR, \LBAND, \LBOR\footnotemark[3].

\bi\ Intractable for e.g. the set of 32-bit integers.

\ \\

\bi\ We may use property-based testing \cite{quickcheck}.

\bi\ Test the property for a sufficient subset of $S$.

\bi\ Seems like the most practical approach.

\bi\ Applicable to the entire language, not just a subset.

\ \\

\bi\ The cost is soundness.

\bi\ The test may generate false positives.

\bi\ We conjecture that a non-associative function will typically be
non-associative for a large subset of $S$.

\footnotetext{Clearly not \LBSL and \LBSR.}

\end{frame}
