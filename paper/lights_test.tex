\newcommand{\XOR}{\mathbin{^\wedge}}
\section{Light's associativity test}
Light's associativity test is an algorithm for testing if a binary operator defined in a
finite set is associative. The idea is to take an element $a$ from the set $S$ with binary
operator $\cdot$ and define two binary operations, $\circ$ and $*$ (see below), construct
their Caley tables, and check if they are the same. It is sufficient to check a proper
generating subset of S\cite{light}.

\begin{align*}
    x \circ y   &= (x \cdot a) \cdot y \\
    x * y       &= x \cdot (a \cdot y) 
\end{align*}

For binary operations only four Caley tables of size $2\times2$ are needed. For example, we
can check the associativity of the operator $x \oplus y = x \XOR  y \mid x \XOR y$:

\makeTable
{assoc-array-combinators}
{Cayley table for the example operator}
{|c|c|c|}
{$\oplus$ & $0$ & $1$}
{
    $0$ & $0$ & $1$\\
    $1$ & $1$ & $0$
}

Because $\{0\}$ is a proper generating subset of $\{0, 1\}$, it suffices to create the following two tables:

\begin{table}[h!]
  \begin{tabular}{|c|c|c|}
    \hline
      $\circ$ & 0 & 1\\
    \hline
      0 & 0 & 1 \\
      1 & 1 & 0 \\
    \hline
  \end{tabular}
\quad
  \begin{tabular}{|c|c|c|}
    \hline
      $*$ & 0 & 1\\
    \hline
      0 & 0 & 1 \\
      1 & 1 & 0 \\
    \hline
  \end{tabular}
  \centering
  \caption{Caley tables for $0$}
\end{table}


