\newcommand{\XOR}{\mathbin{^\wedge}}
\section{Light's associativity test}
Light's associativity test is an algorithm for testing if a binary operator defined in a
finite set is associative. The idea is to take an element $a$ from the set $S$ with binary
operator $\cdot$ and define two binary operations, $\circ$ and $*$ (see below), construct
their Caley tables, and check if they are the same. It is sufficient to check a proper
generating subset of S\cite{light}. 

\begin{align*}
x \circ_{a} y   &= (x \cdot a) \cdot y \\
x *_{a} y       &= x \cdot (a \cdot y) 
\end{align*}

For bitwise operations only four Caley tables of size $2\times2$ are needed. As an
example, we check the associativity of the operator $x \oplus y = x \XOR y \mid x \XOR y$:

\makeTable
{assoc-array-combinators}
{Cayley table for the example operator.}
{|c|c|c|}
{$\oplus$ & $0$ & $1$}
{
    $0$ & $0$ & $1$\\
    $1$ & $1$ & $0$
}

Because $\{0\}$ is a proper generating subset of $\{0, 1\}$, it suffices to create the
following two tables:

\begin{table}[h!]
  \def\arraystretch{1.2}
  \begin{tabular}{|c|cc|}
    \hline
      $\circ_{0}$ & $0$ & $1$\\
    \hline
      $0$ & $0$ & $1$ \\
      $1$ & $1$ & $0$ \\
    \hline
  \end{tabular}
\quad
  \begin{tabular}{|c|cc|}
    \hline
      $*_{0}$ & $0$ & $1$\\
    \hline
      $0$ & $0$ & $1$ \\
      $1$ & $1$ & $0$ \\
    \hline
  \end{tabular}
  \centering
  \caption{Light's associativity test, showing $\oplus$ is associative.}
\end{table}

Because calculations are done on fixed size integers, you could in principle use Light's
associativity test to completely determine associativity. This would, however, be
intractable for practical purposes - for 32bit integers, this would require at least
$2^{32^{2}}$ evaluations of the expression in order to fill a single Caley table, which
would need to be done numerous times.

The technique can be made tractable by turning it into a heuristic which cannot guarantee
that the operator is associative, but instead tries to find a counterexample which proves
that it is not associative. This can be done by taking a sample of random values from the
given integer range and constructing a partial Caley table, which is then used by Light's
test. This approach is inspired by \cite{quickcheck} and similar property-based
testing tools.

As an example, two of the Caley tables for the test of the operator $\texttt{max}(x - y, x \XOR y)$
with 3 randomly generated values in the range $-100$ to $100$ is given below:

\begin{table}[h!]
  \def\arraystretch{1.2}
  \begin{tabular}{|c|ccc|}
    \hline
    $\circ_{-24}$ & $-44$ & $-24$ & $82$ \\
    \hline
        $-44$ & $104$ & $84$  & $110$ \\ 
        $-24$ & $44$  & $24$  & $82$  \\
        $82$  & $150$ & $130$ & $56$  \\
    \hline
  \end{tabular}
\quad
  \begin{tabular}{|c|ccc|}
    \hline
    $*_{-24}$ & $-44$ & $-24$ & $82$ \\
    \hline
        $-44$ & $-24$ & $-44$  & $110$ \\ 
        $-24$ & $-44$  & $-24$  & $82$  \\
        $82$  & $110$ & $82$ & $152$  \\
    \hline
  \end{tabular}
  \centering
  \caption{Example test of a binary operator which shows that it is not associative.}
\end{table}
