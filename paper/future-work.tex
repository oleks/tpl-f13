\section{Future work}

One possible direction of future work is to expand Light's test beyond bitwise
operators. We can make the observation that a binary addition is a ternary
bitwise operator, where for every pair of bits there is a third, carry bit.

We can state addition as a bitwise operation, i.e. $f : \set{0,1} \times
\set{0,1} \times \set{0,1} \rightarrow \set{0,1} \times \set{0,1}$, and
$f\p{x,y,c}=\p{\p{x \oplus y} \oplus c, \p{x \wedge y} \vee \p{ \p{ x \vee y}
\wedge c}}$, where the first element is the result bit, and the second element
is the carry bit. Note, a function like $f$ in L0, using bitwise operators over
integers, is by no means the same as addition over integers.

If we apply the idea of the associative type, we can state the problem of
testing the associativity of integer addition as testing whether for all
$x,y,z,c\in \set{0,1}$:

\[f\p{f\p{x,y,c}, z,c} = f\p{x, f\p{y,z,c},c}.\]

Light's associativity can be applied here by first letting $c=0$ and then
letting $c=1$. If $f$ turns out to be associative in both these cases, we can
say that $f$ is associative in general. 

Problems arise with more than one addition in a single expression, as this requires more
than a single bit for the carry bit, but which might be solvable by expanding the Caley
tables to include the combined carry as an integer.

This approach requires time linear in the number of carry-based operations in
the function body, i.e. there is an independent carry bit for each such
operation. This makes the problem feasible such operations, unlike considering
the entire integer domain.

Subtraction can be dealt with in a similar manner, if we keep in mind that
$x\lp y=x \lp(\lneg y)$. Other operators however, get more tricky as we have to
keep more than a single carry bit in mind when doing e.g. multiplication,
division, raising to various powers, or bit shifting.

Another possible direction of future work is expanding the set of rewriting
rules. Especially useful would be generalised rules about \texttt{min},
\texttt{max}, and \texttt{if-then-else}.

A third, perhaps less useful direction is to generate variable constraints for
unconstrained variables if the function were to be made associative.

If $\oplus\p{x,y,c} = \p{c\lt x} \lp \p{c \lt y}$, and we would like to know
whether $\oplus$ is associative wrt. $x$ and $y$, we reach no useful conclusion
with the above method. We can find out whether there exists a $c$ for which
this property holds:

\begin{align*}
\p{\p{c \lt c} \lt x} \lp \p{c \lt z} &= \p{c \lt x} \lp \p{\p{c \lt c} \lt z} \\
\p{c \lt \p{c \lt x}} \lp \p{c \lt z} &= \p{c \lt x} \lp \p{c \lt \p{c \lt z}} \\
\p{c \lt x} \lp z &= x \lp \p{c \lt z} \\
\p{c \lt x} \lm \p{c \lt z} &= x \lm z \\
c \lt \p{x \lm z} &= x \lm z \\
c &= \p{x \lm z} \ld  \p{x \lm z} \\
c &= 1
\end{align*}

The compiler could at this point suggest that $c$ be replaced by a constant if
the function is to be associative. Such an analysis may reveal a constant, a
range, or a set of possible values, or it may fail completely.

