\section{Future work}

One possible direction of future work is to expand Light's test beyond bitwise
operators. We can make the observation that a binary addition is a ternary
bitwise operator, where for every pair of bits there is a third, carry bit.

If we apply the idea of our associative type, we can state the problem of
testing the associativity of integer addition as testing whether $f\p{x,y,c}$
is associative, i.e. whether for all $x,y,z,c\in \set{0,1}$:

\[f\p{f\p{x,y,c}, z,c} = f\p{x, f\p{y,z,c},c}.\]

Light's associativity can be applied here by first letting $c=0$ and then
letting $c=1$. If $f$ turns out to be associative in both these cases, we can
say that $f$ is associative in general.

This approach requires time linear in the number of carry-based operations in
the function body, i.e. there is an independent carry bit for every addition.

Subtraction can be dealt with in a similar way, if we keep in mind that $x\lp
y=x \lp(\lneg y)$. Other operators however, get more tricky as we have to keep
more than a single carry bit in mind when doing multiplication, division,
raising to various powers, bit shifting, etc.

Another possible direction of future work is expading the set of rewriting
rules. Especially useful would be generalised rules about \texttt{min},
\texttt{max}, and \texttt{if-then-else}.

