\section{Decider algorithm}
In order to get a decider that works well in practice, the methods presented in this paper
needs to be combined such that each method handles the domains where it works. In this
section we will breifly discuss the advantages and drawbacks of each method and in which
cases we believe they should be used.

Note that since none of our methods currently have a practical implementation we cannot
say anything about the effectiveness of the different methods in practice.

The simple examples can be decided completely, either by the chaining tables, or, if the
expressions consist of combinations of bitwise operators, by Light's associativity test.
Since the binary operators work on arrays, for arrays of known length it is possible to
decide associativity on an per index basis - for example, these methods could determine
that index 1, 2, and 4 were associative for certain, while they couldn't say anything
about index 3.

If the expressions are mixed and only contain operators with known rewrite rules, they can
be handled by the rewrite algorithms described in Section \ref{section:rewriting}. This
method is sound, and returns either that the expression is associative or that it cannot
decide.

If none of the other methods have returned with certainty or been applicable, Light's test
can be applied as described at the end of Section \ref{section:lights-test}. It cannot
decide if the expression is associative, but instead tries to disprove it by taking random
values from the functions input type. 
