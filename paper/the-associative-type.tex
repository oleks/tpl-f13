\section{The associative type}\label{section:the-associative-type}

For ease of analysis, we introduce the concept of the \emph{associative type}
--- a function type that unifies with the type of any associative function. We
identify this type from the following observation: an associative function
consumes at least two values of some type $S$, and returns a value of type $S$.

We define it formally as follows:

\[\oplus:S \boxtimes S \boxtimes T_1 \boxtimes T_2 \boxtimes \cdots \boxtimes
T_n \rightarrow S, \]

where

\begin{itemize}

\item $\p{T_k}_{k=1}^n$ for $n\geq 0$ is a sequence of arbitrary types.

\item $\boxtimes$ is the commutative equivalent of the regular $\times$, i.e.
the arguments to $\oplus$ can be rearranged under unification.

\end{itemize}

For example, the following types unify with the associative type: $f:S \times S
\times S \rightarrow S$, $g:S \times S \times \mathcal{Z} \rightarrow S$, $h:S
\times \mathcal{Z} \times S \rightarrow S$, etc.

A function with a type that unifies with the associative type is not
necessarily associative, but a function that doesn't definitely is not.
