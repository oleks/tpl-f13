\section{Builtins}

\subsection{Chaining}

We consider first whether simply chaining applications of the operator is
associative. That is, for each binary operator $\oplus:t\times t \rightarrow
t$, we consider whether $\p{a \oplus b} \oplus c = a \oplus \p{ b \oplus c }$
by the L0 specification. Unary operators are trivially associative under
chaining. For binary operators with arguments of two different types, as well
as ternary and n-ary operators, it depends. Later, we'll allow intermixing of
operators.

\subsubsection{Arithmetic operators}

L0 has a range of binary arithmetic operators, namely \texttt{+}, \texttt{*},
\texttt{-}, \texttt{/}, \texttt{pow}, and \tilde, where the last operator is a
unary negation operator. The operators defined on arguments of type either
\intt{} or \realt{} (L0 is strongly-typed), yielding a value of the same type.

As of now, there are no restrictions in the L0 standard as to how \realt{}s are
to be represented, so we confine our attention to \intt{}s. For \intt{}s we
assume a $p$-bit two's complement representation.

Our observations are summarized in \referToTable{assoc-chaining-arithop}.

\makeTable
{assoc-chaining-arithop}
{Associativity of chaining arithmetic operators.}
{|c|c|}
{\textbf{Operator} & \textbf{Chaining is associative}}
{
  \texttt{+}   & yes \\
  \texttt{*}   & yes \\
  \texttt{-}   & no \\
  \texttt{\%}  & no \\
  \texttt{pow} & no \\
  \tilde       & yes\footnotemark[1]
}

\subsubsection{Relational operators}

L0 has a couple binary relational operators, namely \texttt{=}, \texttt{<}, and
\texttt{<=}, having the obvious semantics. The operators are defined on
arguments both of type either \intt{} or \realt{}, yielding a value
of type \boolt{}.

Relational operators are not chainable in L0. These operators are only
interesting in combination with other operators.

\subsubsection{Logical operators}

L0 has a couple logical operators, namely \texttt{\&\&}, \texttt{||}, and
\texttt{not}, where the last is a unary negation operator. The operators are
defined on arguments of type \boolt{}, yielding a value of type \boolt{}.

Our observations are summarized in \referToTable{assoc-chaining-logop}.

\makeTable
{assoc-chaining-logop}
{Associativity of chaining logical operators.}
{|c|c|}
{\textbf{Operator} & \textbf{Chaining is associative}}
{
  \texttt{\&\&} & yes \\
  \texttt{||}   & yes \\
  \texttt{not}  & yes\footnotemark[1]
}

\footnotetext[1]{All unary operators are associative by definition.}

\subsubsection{Bitwise operators}

L0 has a range of bitwise operators, namely \texttt{\^}, \texttt{\&},
\texttt{|}, \texttt{>{}>}, and \texttt{<{}<}, having semantics as in C. The
operators are defined on arguments of type \intt{}, yielding a value of type
\intt{}.

Our observations are summarized in \referToTable{assoc-chaining-bitop}.

\makeTable
{assoc-chaining-bitop}
{Associativity of chaining bitwise operators.}
{|c|c|}
{\textbf{Operator} & \textbf{Chaining is associative}}
{
  \texttt{\^} & yes \\
  \texttt{\&} & yes \\
  \texttt{|}  & yes \\
  \texttt{>{}>} & yes \\
  \texttt{<{}<} & yes
}

\subsubsection{Other functions}

L0 has a range of other builtins, e.g. various second-order array combinators.
Of particular interest to us are the functions \texttt{concat}, \texttt{min},
and \texttt{max}. The other builtins do not meet the requirement of
unifiability of their type with an associative type, or fall prey to the fact
that functions are not first-class citizens in L0. These operators are
trivially associative.

Our observations are summarized in \referToTable{assoc-array-combinators}.

\makeTable
{assoc-array-combinators}
{Associativity of second-order array combinators}
{|c|c|}
{\textbf{Operator} & \textbf{Chaining is associative}}
{
  \texttt{concat} & yes \\
  \texttt{min} & yes \\
  \texttt{max}  & yes
}

\subsection{Combining}

Light's associativity test\cite{light-mathematica}.

