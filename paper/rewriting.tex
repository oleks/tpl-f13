% L0 is a pure language

\section{Rewriting}

The straight-forward way of testing associativity is by \emph{rewriting}. We
can test whether $\oplus$ is associative by considering whether $\p{x \oplus y}
\oplus z$ and $x \oplus \p{y \oplus z}$ can be rewritten to be syntactically
equivalent.

For instance, if $\oplus\p{x,y} = x\cdot c + y \cdot c$, for some constant
$c\in S$, we consider whether the following equality holds for all $x$, $y$,
and $z$:

\begin{align*}
\p{x\cdot c + y\cdot c}\cdot c + z\cdot c &= x\cdot c + \p{y\cdot c + z\cdot c}\cdot c \\
x\cdot c \cdot c + y\cdot c \cdot c + z\cdot c &= x\cdot c + y\cdot c \cdot c + z\cdot c \cdot c \\
x\cdot c \cdot c + z\cdot c &= x\cdot c + z\cdot c \cdot c
\end{align*}

This analysis required \emph{rewriting} both expressions as a \emph{sum of
products}, followed by \emph{common term elimination}. If $c$ is a constant, we
can complete the analysis by \emph{constant folding}, and considering whether
the left-hand side is syntactically equivalent to the right-hand side.

Note, we didn't need to utilize the properties that addition and multiplication
are also commutative.  This is because the variables appear in the same order
on either side of the equality. So long as our rewriting rules respect this
order, we can proceed freely. 

Another property that our rewriting rules must respect are overflows. We should
keep in mind that we're dealing with modulo arithmetic.

\subsection{Generating variable constraints}

If instead, $\oplus\p{x,y,c} = x\cdot c + y \cdot c$, and we would like to know
whether $\oplus$ is associative wrt. $x$ and $y$, we reach no useful
conclusion. We can find out whether there exists a $c$ for which this property
holds:

\begin{align*}
x \cdot c^2 + z \cdot c &= x \cdot c + z \cdot c^2 \\
x \cdot c + z &= x + z \cdot c \\
x \cdot c - z \cdot c &= x - z \\
c (x-z) &= x-z \\
c &= {x-z\over x-z} \\
c &= 1
\end{align*}

The compiler should at this point suggest that $c$ be replaced by a constant if
the function is to be associative. Such an analysis may not reveal a constant, 

This implies that $\oplus$ is associative for all $x$, $y$, and $z$ only if
$\card{c}=1$. Of course, if $c$ is a constant, it is known to us, and the
equality conclusion can be reached directly without solving for $c$.

To reach this
conclusion, we rewrote both expressions as a sum of products, and found a $c$
such that the left-hand side is syntactically equivalent to the right-hand side
after constant folding.

This technique requires the rewriting of arbitrary program text into some
normal form that subsequently allows equality comparison of program text.

Common subexpression elimination
