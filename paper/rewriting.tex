% L0 is a pure language

\section{Rewriting}\label{section:rewriting}

The straight-forward way of testing associativity is by \emph{rewriting}. We
can test whether $\oplus$ is associative by considering whether $\p{x \oplus y}
\oplus z$ and $x \oplus \p{y \oplus z}$ can be rewritten to be syntactically
equivalent.

For instance, if $\oplus\p{x,y} = \p{c \lt x} \lp \p{c \lt y}$, for some
constant $c\in S$, we consider whether the equality in
\referToFigure{first-equality} holds for all $x$, $y$, and $z$.

\begin{figure*}[htbp!]
\begin{align*}
\p{c \lt \p{\p{c\lt x} \lp \p{c\lt y}}} \lp \p{c \lt z} &= \p{c\lt x} \lp \p{c
\lt \p{\p{c\lt y} \lp \p{c\lt z}}} \\
\p{\p{c\lt \p{c \lt x}} \lp \p{c\lt \p{c \lt y}}} \lp \p{c\lt z} &= \p{c\lt x} \lp
\p{\p{c\lt \p{c \lt y}} \lp \p{c\lt \p{c \lt z}}} \\
\p{\p{c\lt \p{c \lt x}} \lp \p{c\lt \p{c \lt y}}} \lp \p{c\lt z} &= \p{\p{c\lt
x} \lp \p{c\lt \p{c \lt y}}} \lp \p{c\lt \p{c \lt z}} \\
\p{\p{c\lt c} \lt x} \lp \p{\p{c\lt c} \lt y} \lp \p{c\lt z} &= \p{\p{c\lt
x} \lp \p{\p{c\lt c} \lt y}} \lp \p{\p{c\lt c} \lt z} \\
\p{\p{c\lt c} \lt x} \lp \p{c\lt z} &= \p{c\lt x} \lp \p{\p{c\lt c} \lt z}
\end{align*}
\caption[]{If $\oplus\p{x,y}=\p{c\lt x} \lp \p{c\lt y}$ is associative, the
above should hold for all $x$, $y$ and $z$.}
\label{figure:first-equality}
\end{figure*}

This analysis requires \emph{rewriting} both expressions as a \emph{sum of
products}, followed by \emph{common term elimination}. If $c$ is a constant, we
can complete the analysis by \emph{constant folding}, and considering whether
the left-hand side is syntactically equivalent to the right-hand side.

Note, we didn't need to utilize the properties that addition and multiplication
are also commutative.  This is because the variables appear in the same order
on either side of the equality. So long as our rewriting rules respect this
order, we can proceed freely. 

Another property that our rewriting rules must respect are overflows. We should
keep in mind that we're dealing with modulo arithmetic. Rewriting should not
change the evaluation of any expression with overflow.

\subsection{Algorithm}

We now specify the overall algorithm:

\begin{enumerate}

\item Let $f(x,y)=e$.

\item Let $z$ be a variable name that does not appear in $e$.

\item Let $e_1$ be a copy of $e$.

\item Let $e_2$ be a copy of $e$, where all occurrences of $y$ have been
replaced by $z$, and all occurrences of $x$ have been replaced by the program
text $\p{e_1}$ (in that order). Add all occurrences to $Q$.

\item Let $e_3$ be a copy of $e$, where all occurrences of $y$ have been
replaced by $z$, and all occurrences of $x$ have been replaced by $y$ (in that
order).

\item Let $e_4$ be a copy of $e$, where all occurrences of $y$ have been
replaced by the program text $\p{e_3}$. Add all occurrences to $Q$.

\item Use $Q$ to determine whether $e_2$ and $e_4$ are semantically equivalent.

\end{enumerate}

For instance, if $f(x,y)=\p{c\lt x} \lp \p{c\lt y}$:

\begin{align*}
e_1 &= \p{c\lt x} \lp \p{c\lt y} \\
e_2 &= \underbrace{\p{c \lt \p{\p{c\lt x} \lp \p{c\lt y}}}}_{q_1} \lp \p{c \lt
z} \\
e_3 &= \p{c\lt y} \lp \p{c \lt z} \\
e_4 &= \p{c \lt x} \lp \underbrace{\p{c \lt \p{ \p{c \lt y} \lp \p{c \lt z}
}}}_{q_2} \\
Q &= \seq{q_1, q_2 }
\end{align*}

$Q$ is a working queue which we will use to attempt to rewrite $e_2$ and $e_4$
into syntactically equivalent expressions. We proceed to rewrite the
subexpressions $q_1$ and $q_2$ by applying various rewrite rules. A rewrite
rule will typically introduce new parentheses, which designate monotonically
nonincreasing subexpressions to be rewritten next.

For instance, from the rewrite rules in \referToFigure{rewriting}, we see that
for $q_1$, in the example above, we apply the rule $\proc{T-Plus-L}$. This
rewrites $q_1$ into

\[\underbrace{\p{c \lt \p{c \lt x}}}_{q_3} \lp \underbrace{\p{c \lt \p{c \lt
y}}}_{q_4},\]

where $q_3$ and $q_4$ get queued for rewriting. $q_3$ and $q_4$ both fall prey
to the associative chaining property of multiplication.
\referToFigure{rewriting-chaining} indicates that we rewrite such chains to be
left-associative. If no rewrite rules match a subexpression on the queue, the
subexpression is simply popped from the queue.

The algorithm terminates if there are no circular rewrites. This can be checked
by building a \emph{rewrite rule graph} where the nodes are the rewrite rules.
There is a directed edge from one rewrite rule to another if the rewrite yields
an expression rewritable by the other. If there are no cycles in this graph,
the finite nature of syntax trees of finite programs makes the algorithm
terminate.

\subsection{Rules}

The soundness of the algorithm depends on the soundness of the rewrite rules.
The incompleteness of the method was proven in \referToSection{undecidability}. 

In general, the rewrite rules are based on the observations made in
\referToSection{chaining}, as well as the distributive laws of various
operators. We keep the algorithm deterministic such that there is at most one
edge going out of any rewrite rule in the rewrite rule graph.

An operator $\otimes:S\times S\rightarrow S$ is distributive over
$\oplus:S\times S \rightarrow S$ iff $x \otimes \p{ y \oplus z } = \p{x \otimes
y} \oplus \p{x \otimes z}$(it is left-distributive) and $\p{y \oplus z} \otimes
x = \p{y \otimes x} \oplus \p{z \otimes x}$(it is right-distributive).

First, there are the rewrite rules that follow directly from
\referToSection{chaining} in \referToFigure{rewriting-chaining}. We specify
only their left-associative parts because otherwise this would lead to cycles
in the rewrite rule graph. We chose to rewrite all chains into left-associative
chains as most operators in L0 are left-associative. We hypothesise this will
make the recursion bottom out faster in practice.

Second, we have the rules for various arithmetic, logical, and bitwise
operators in \referToFigure{rewriting}. To construct the rules we proceed as
follows: for every operator $\otimes$, consider every other operator $\oplus$,
as to whether $\otimes$ is distributive over $\oplus$.

Similar rules can be written for \texttt{min}, \texttt{max}, and
\texttt{if-then-else}. \texttt{concat} however does not distribute over any of
the operators.

%For \LMIN{} and \LMAX{} the situation is a little more complicated (\LCONCAT{}
%is not distributtive over any of th other operators covered in
%\referToSection{chaining}. We have not come up with any sufficiently general
%rewrite rule. We have come up with one rule presented in
%\referToFigure{rewriting-min-max}.

We do not claim that this list of rewrite rules is complete in any way. Our
approach allows for rules to be added once they are discovered.

\begin{figure*}[htbp!]
\[
\begin{matrix}
\proc{Plus}:{\over
e \lp \p{e_1 \lp e_2} \rightarrow \p{ e \lp e_1} \lp e_2
}\quad
\proc{Times}:{\over
e \lt \p{e_1\lt e_2} \rightarrow \p{e \lt e_1} \lt e_2
}\\
\proc{Land}:{\over
e \LAND \p{e_1 \LAND e_2} \rightarrow \p{ e \LAND e_1} \LAND e_2
}\quad
\proc{Lor}:{\over
e \LOR \p{e_1\LOR e_2} \rightarrow \p{e \LOR e_1} \LOR e_2
}\\
\proc{Xor}:{\over
e \LBXOR \p{e_1 \LBXOR e_2} \rightarrow \p{ e \LBXOR e_1} \LBXOR e_2
}\\
\proc{Band}:{\over
e \LBAND \p{e_1 \LAND e_2} \rightarrow \p{ e \LBAND e_1} \LBAND e_2
}\quad
\proc{Bor}:{\over
e \LBOR \p{e_1\LBOR e_2} \rightarrow \p{e \LBOR e_1} \LBOR e_2
}\\
\proc{BSL}:{\over
e \LBSL \p{e_1 \LBSL e_2} \rightarrow \p{ e \LBSL e_1} \LBSL e_2
}\quad
\proc{BSR}:{\over
e \LBSR \p{e_1\LBSR e_2} \rightarrow \p{e \LBSR e_1} \LBSR e_2
}\\
\proc{Min}:{\over
\LMIN\p{e, \LMIN\p{e_1, e_2}} \rightarrow \LMIN\p{\LMIN\p{ e, e_1}, e_2}
}\\
\proc{Min}:{\over
\LMAX\p{e, \LMAX\p{e_1, e_2}} \rightarrow \LMAX\p{\LMAX\p{ e, e_1}, e_2}
}\\
\proc{Concat}:{\over
\LCONCAT\p{e, \LCONCAT\p{e_1, e_2}} \rightarrow \LCONCAT\p{\LCONCAT\p{ e, e_1}, e_2}
}\\
\end{matrix}
\]
\caption[]{Rewriting rules for operators that are associative under chaining.}
\label{figure:rewriting-chaining}
\end{figure*}

\begin{figure*}[htbp!]
\[
\begin{matrix}
\proc{P-Minus-L}:{\over
e \lp \p{e_1 \lm e_2} \rightarrow \p{e \lp e_1 } \lm e_2
}\\
\proc{P-Min-L}:{\over
e \lp \LMIN\p{e_1, e_2} \rightarrow \LMIN\p{\p{e \lp e_1}, \p{e \lp e_2}}
}\quad
\proc{P-Max-L}:{\over
e \lp \LMAX\p{e_1, e_2} \rightarrow \LMAX\p{\p{e \lp e_1}, \p{e \lp e_2}}
}\\
\proc{T-Plus-L}:{\over
e \lt \p{e_1 \lp e_2} \rightarrow \p{ e \lt e_1 } \lp \p{ e \lt e_2 }
}\quad
\proc{T-Minus-L}:{\over
e \lt \p{e_1 \lm e_2} \rightarrow \p{ e\lt e_1 } \lm \p{ e \lt e_2 }
}\\
\proc{T-Mod-L}:{\over
e \lt \p{e_1 \lmod e_2}\rightarrow \p{ e \lt e_1 } \lmod \p{ e \lt e_2 }
}\quad
\proc{T-ShiftL-L}:{\over
e \lt \p{e_1 \LBSL e_2} \rightarrow \p{ e \lt e_1 } \LBSL e_2
}\\
\proc{T-Min-L}:{\over
e \lt \LMIN\p{e_1, e_2} \rightarrow \LMIN\p{\p{e \lt e_1}, \p{e \lt e_2}}
}\quad
\proc{T-Max-L}:{\over
e \lt \LMAX\p{e_1, e_2} \rightarrow \LMAX\p{\p{e \lt e_1}, \p{e \lt e_2}}
}\\
\proc{A-Or-L}:{\over
e \LAND \p{e_1 \LOR e_2}\rightarrow \p{ e \LAND e_1 } \LOR \p{ e \LAND e_2 }
}\quad
\proc{BA-BOr-L}:{\over
e \LBAND \p{e_1 \LBOR e_2}\rightarrow \p{ e \LBAND e_1 } \LBOR \p{ e \LBAND e_2
}
}
\end{matrix}
\]
\caption[]{Rewriting rules for arithmetic, logical, and bitwise operators. The
above rules cover only left-distributive laws. Right-distributive laws have
symmetrical rewrite rules.  Note, bit shifting to the left by $e$ is equivalent
to multiplying by $2^e$.}
\label{figure:rewriting}
\end{figure*}

%\begin{figure*}[htbp!]
%\[
%\begin{matrix}
%\proc{If-L}:{\over
%e \circ \p{\mathtt{if}\ c\ \mathtt{then}\ e_1\ \mathtt{else}\ e_2} \rightarrow
%\mathtt{if}\ c\ \mathtt{then}\ \p{e \circ e_1} \ \mathtt{else}\ \p{e \circ e_2}
%}\\
%\end{matrix}
%\]
%\caption[]{Rewriting rules for \texttt{if}-\texttt{then}-\texttt{else}
%expressions. $\circ$ is any operator in L0.}
%\label{figure:rewriting-arithop}
%\end{figure*}


%\begin{figure*}[htbp!]

%\begin{align*}
%\proc{P-PlusR}:{\over
%\p{e_1 \lp e_2 \lp \cdots \lp e_n} \lp e \rightarrow e_1 \lp e_2 \lp \cdots \lp e_n \lp e
%}\p{n\geq 2}\\
%\proc{P-PlusL}:{\over
%e \lp \p{e_1 \lp e_2 \lp \cdots \lp e_n} \rightarrow e \lp e_1 \lp e_2 \lp \cdots \lp e_n
%}\p{n\geq 2}
%\end{align*}

%\begin{align*}
%\proc{P-TimesR}:{\over
%\p{e_1 \lp e_2 \lp \cdots \lp e_n} \lt e \rightarrow e_1\lt e \lp e_2\lt e \lp
%\cdots \lp e_n \lt e
%}\p{n\geq 2}\\
%\proc{P-TimesL}:{\over
%e\lt\p{e_1 \lp e_2 \lp \cdots \lp e_n} \rightarrow e\lt e_1 \lp e\lt e_2 \lp \cdots
%\lp e\lt e_n
%}\p{n\geq 2}
%\end{align*}

%\begin{align*}
%\proc{Minus}:{\over
%e_1 \lm e_2 \rightarrow e_1 \lp \lneg e_2
%}
%\end{align*}

%\begin{align*}
%\proc{P-Negate}:{\over
%\lneg\p{e_1 \lp e_2 \lp \cdots \lp e_n} \rightarrow \lneg e_1 \lp \lneg e_2 \lp \cdots
%\lp \lneg e_n
%}\p{n\geq 2}
%\end{align*}


%\caption[]{Rewriting rules for arithmetic operators.}
%\label{figure:rewriting-arithop}
%\end{figure*}


%\begin{figure*}[htbp!]
%\begin{align*}
%\proc{P-AndR}:{\over
%\p{e_1 \LOR e_2 \LOR \cdots \LOR e_n} \LAND e \rightarrow e_1\LAND e \LOR
%e_2\LAND e \LOR \cdots \LOR e_n \LAND e
%}\p{n\geq 2}\\
%\proc{P-AndL}:{\over
%e\LAND\p{e_1 \LOR e_2 \LOR \cdots \LOR e_n} \rightarrow e\LAND e_1 \LOR e\LAND
%e_2 \LOR \cdots
%\LOR e\LAND e_n
%}\p{n\geq 2}
%\end{align*}

%\begin{align*}
%\proc{P-OrR}:{\over
%\p{e_1 \LOR e_2 \LOR \cdots \LOR e_n} \LOR e \rightarrow e_1 \LOR e_2 \LOR
%\cdots \LOR e_n \LOR e
%}\p{n\geq 2}\\
%\proc{P-OrL}:{\over
%e \LOR \p{e_1 \LOR e_2 \LOR \cdots \LOR e_n} \rightarrow e \LOR e_1 \LOR e_2
%\LOR \cdots \LOR e_n
%}\p{n\geq 2}
%\end{align*}

%\begin{align*}
%\proc{P-Not}:{\over
%\LNOT\p{e_1 \LOR e_2 \LOR \cdots \LOR e_n} \rightarrow \LNOT e_1 \LOR \LNOT e_2
%\LOR \cdots \LOR \LNOT e_n
%}\p{n\geq 2}
%\end{align*}

%\caption[]{Rewriting rules for logical operators.}
%\label{figure:rewriting-logop}
%\end{figure*}

%\begin{figure*}[htbp!]
%\begin{align*}
%\proc{S-Min}:{\over \LMIN\p{e_y, e_x} \rightarrow \LMIN\p{e_x,e_y} }
%\end{align*}

%\begin{align*}
%\proc{P-MinL}:{\over \LMIN\p{e_x, e_1 \cdots \LMIN\p{e_y, e_z} \cdots e_n}
%\rightarrow \LMIN \p{ \LMIN \p{ e_x, e_1 \cdots e_y \cdots e_n}, e_1 \cdots e_z
%\cdots e_n }
%}\p{n\geq 0}
%\end{align*}

%\caption[]{Rewriting rules for \texttt{min}. Rewrite rules for \texttt{max} are
%symmetrical. $e_x$, $e_y$, and $e_z$ represent expressions that contain a
%variable $x$, $y$, and $z$, respectively.  The sequences $\p{e_k}_{k=1}^n$ are
%independent of $e_x$, $e_y$, and $e_z$.} \label{figure:rewriting-logop}
%\end{figure*}

%\begin{table}
%\centering
%\begin{tabular}{|l|l|l|}
%\hline
%\textbf{Precedence} & \textbf{Operators} & \textbf{Fixity} \\ \hline
%1 & \texttt{not}, \tilde & None \\\hline
%2 & \texttt{pow} & Left \\\hline
%3 & \texttt{*}, \texttt{/}, \texttt{\%} & Left \\\hline
%4 & \texttt{+}, \texttt{-} & Left \\\hline
%5 & \texttt{<{}<{}}, \texttt{>{}>{}} & Left \\\hline
%6 & \texttt{<=}, \texttt{<}, \texttt{=} & None\footnotemark[2] \\\hline
%7 & \texttt{\&}, \texttt{\^}, \texttt{|} & Left \\\hline
%8 & \texttt{\&\&} & Left \\\hline
%9 & \texttt{||} & Left \\\hline
%10 & \texttt{if} & None \\\hline
%\end{tabular}
%\caption[]{The precedence and fixity of operators in L0. Lower precedence means
%stronger binding.}
%\label{table:precedence}
%\end{table}

