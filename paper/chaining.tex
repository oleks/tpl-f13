\section{Chaining}

We consider first whether simply chaining applications of an operator is
associative. That is, for each binary operator $\oplus:S\times S \rightarrow
S$, we consider whether $\forall\ x,y,z\in S\ .\ \p{x \oplus y} \oplus z = x
\oplus \p{y \oplus z}$ by the L0 specification.

Unary operators are trivially associative under chaining. For binary operators
with arguments of two different types, as well as ternary and n-ary operators,
it depends. Later, we'll allow intermixing of operators.

Note that even though the binary functions in L0 we are interested in work on arrays, we
will mostly focus on the associativity of binary operators working on single values, and
thus the examples in the text will predominantly be integer based.

\subsubsection{Arithmetic operators}

L0 has a range of binary arithmetic operators, namely \texttt{+}, \texttt{*},
\texttt{-}, \texttt{/}, \texttt{\%}, \texttt{pow}, and \tilde, where the last
operator is a unary negation operator. The operators defined on arguments of
type either \intt{} or \realt{} (L0 is strongly-typed), yielding a value of the
same type.

As of now, there are no restrictions in the L0 standard as to how \realt{}s are
to be represented, so we confine our attention to \intt{}s. For \intt{}s we
assume a $p$-bit two's complement representation.

Our observations are summarized in \referToTable{assoc-chaining-arithop}.

\makeTable
{assoc-chaining-arithop}
{Associativity of chaining arithmetic operators.}
{|c|c|}
{\textbf{Operator} & \textbf{Chaining is associative}}
{
  \texttt{+}   & yes \\
  \texttt{*}   & yes \\
  \texttt{-}   & no \\
  \texttt{/}   & no \\
  \texttt{\%}  & no \\
  \texttt{pow} & no \\
  \tilde       & yes\footnotemark[1]
}

\subsubsection{Relational operators}

L0 has a couple binary relational operators, namely \texttt{<=}, \texttt{<}, and
\texttt{=}, having the obvious semantics. The operators are defined on
arguments both of type either \intt{} or \realt{}, yielding a value
of type \boolt{}.

Relational operators are not chainable in L0. These operators are only
interesting in combination with other operators.\footnotemark[2]

\subsubsection{Logical operators}

L0 has a couple logical operators, namely \texttt{\&\&}, \texttt{||}, and
\texttt{not}, where the last is a unary negation operator. The operators are
defined on arguments of type \boolt{}, yielding a value of type \boolt{}.

Our observations are summarized in \referToTable{assoc-chaining-logop}.

\makeTable
{assoc-chaining-logop}
{Associativity of chaining logical operators.}
{|c|c|}
{\textbf{Operator} & \textbf{Chaining is associative}}
{
  \texttt{\&\&} & yes \\
  \texttt{||}   & yes \\
  \texttt{not}  & yes\footnotemark[1]
}

\footnotetext[1]{Unary operators can be regarded as associative by definition.}

\footnotetext[2]{This is no longer true at the time of our submission.
\texttt{<=}, \texttt{<}, and \texttt{=} are now arithemtic operators, all
left-fixed, and returning integral values.}

\subsubsection{Bitwise operators}

L0 has a range of bitwise operators, namely \texttt{\^}, \texttt{\&},
\texttt{|}, \texttt{>{}>}, and \texttt{<{}<}, having semantics as in C. The
operators are defined on arguments of type \intt{}, yielding a value of type
\intt{}.

Our observations are summarized in \referToTable{assoc-chaining-bitop}.

\makeTable
{assoc-chaining-bitop}
{Associativity of chaining bitwise operators.}
{|c|c|}
{\textbf{Operator} & \textbf{Chaining is associative}}
{
  \texttt{\^} & yes \\
  \texttt{\&} & yes \\
  \texttt{|}  & yes \\
  \texttt{>{}>} & yes \\
  \texttt{<{}<} & yes
}

\subsubsection{Other functions}

L0 has a range of other builtins, e.g. various second-order array combinators.
Of particular interest to us are the functions \texttt{concat}, \texttt{min},
and \texttt{max}, as they are clearly associative.

The other builtins do not meet the requirement of unifiability of their type
with an associative type, or fall prey to the fact that functions are not
first-class citizens in L0.

Our observations are summarized in \referToTable{assoc-array-combinators}.

\makeTable
{assoc-array-combinators}
{Associativity of second-order array combinators}
{|c|c|}
{\textbf{Operator} & \textbf{Chaining is associative}}
{
  \texttt{concat} & yes \\
  \texttt{min} & yes \\
  \texttt{max}  & yes
}
