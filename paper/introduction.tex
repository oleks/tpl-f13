\section{Introduction}

Some parallelizing compilers will assume that the user has supplied an
associative function. Others will require for the function to be explicitly
marked as ``associative''. Others still will try to deduce the associativity
property, and warn the user if the function clearly isn't associative.

\subsection{An associative type}

For ease of analysis, we introduce the concept of an \emph{associative type}
--- a function type that unifies with the type of any associative function. An
associative function consumes at least two values of type $t$, and returns a
value of type $t$.

We refer to functions that unify with the associative type, but are not
themselves associative as \emph{not associative}. We refer to all other
functions as \emph{associative}.

In particular, we refer to functions that do not unify with an associative type
as associative. This is to allow us to say that e.g. unary functions are
associative. Although this does not make sense mathematically, our reasoning is
that functions that do not unify with an associative type cannot be chained
into a function whos type unifies with an associative type.

